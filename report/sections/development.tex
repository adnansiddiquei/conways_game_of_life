%! Author = adnansiddiquei
%! Date = 08/03/2024

\section{Development}\label{sec:development}
    This section discusses the development of the code and software engineering best practises.

    CMake was used to manage the entire build process, including the compilation of the code, and running of tests.
    For formatting, the code was run through \inlinecode{clang-format} automatically on every build to ensure a consistent
    style.
    This was done by writing a custom CMake function and tying it into the build process.

    Tests were written using the GoogleTest framework, and tied into the build process using CMake.
    These tests were run regularly, primarily before creating any merge requests into main.
    To make writing the tests easy, and to ensure that the tests were comprehensive, the code was written in a modular
    way, such that the code was easy to test in small isolated units.

    Generally, best practises with regard to Git were followed, such as creating a new branch for every new feature
    or bug fix, and writing meaningful commit messages.
    New issues were created for every feature, and branches and merge requests were made against these issues to make
    work easy to track.

    Additionally, the code was documented using Doxygen to ensure that the code was easy to understand and maintain.
    This was particularly important as aspects of the code were quite complex.

    Polymrphism was also used to make the code more modular and easier to maintain.
    Much of the utility code is written in a generic manner and then inherited by the specific classes that need it.
    For example, the \inlinecode{Array2D} acted as a base class for the \inlinecode{Array2DWithHalo} which further
    acted as a base class for \inlinecode{ConwaysArray2DWithHalo}.
    This allowed for the code to be kept minimal and reusable.

    Effective error handling was also implemented, raising descriptive errors where possible, especially against
    invalid user input.

